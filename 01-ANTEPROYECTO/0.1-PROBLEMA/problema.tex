% Describa el problema 
[Búsqueda de información que permita construir un marco evolutivo de una línea de tiempo (time line) del problema, detectar las dificultades, aspectos negativos, malestar de entorno, fricciones y todo lo que genera conflictos tanto en la variable tecnológica y la variable del proceso a ser transformado]

Describa de forma directa cual es el motivo actual para plantear el problema, que motivos de entorno tanto en la variable tecnológica y la variable del proceso a ser transformado, delimite en tiempo y espacio el problema]

[Mediante una representación gráfica describir el problema central, sus cusas y efectos del problema, utilice el método de árbol de problemas y/o diagrama de Ishikawa, conocido también como diagrama de espina de pescado, además, puede utilizar otros diagramas como el de causa-efecto, diagrama de grandal o diagrama causal]

Utilice alguna herramienta para diagramar el planteamiento del problema [lucidchart, canva, draw, visio u otra]
